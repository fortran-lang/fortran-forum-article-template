%%  Fortran Forum Latex commands

\documentclass[10pt,letterpaper,twoside]{article}

\usepackage{fullpage}
\usepackage{fancyhdr}
\usepackage{longtable}

\pagestyle{fancy}

\fancyhead{}
\fancyfoot{}

%% 2018 37 April     1
%% 2019 38 August    2
%% 2020 39 December  3

\fancyfoot[OR,EL]{ACM \textit{Fortran Forum}, April 2020, \textbf{39}, 1}
\fancyfoot[C]{\small{\thepage}}

\setlength\textheight{230mm}
\setlength\textwidth{158mm}
\setlength\footskip{16mm}

\setcounter{page}{28}
\setcounter{section}{3}
\setlength\headheight{10pt}

\renewcommand\headrule{}

\makeatletter%

\renewcommand\maketitle{%
    \newpage
    \null
    \vskip 1em
    \begin{center}
    {\Large\sf\bfseries \@title \par}
    \vspace*{6pt}
    {\normalfont\normalsize \@author}
    \end{center}}

\renewcommand\section{\@startsection {section}{1}{\z@}%
                                   {-3.5ex \@plus -1ex \@minus -.2ex}%
                                   {2.3ex \@plus.2ex}%
                                   {\sf\large\bfseries}}

\renewcommand\subsection{\@startsection{subsection}{2}{\z@}%
                                     {-3.25ex\@plus -1ex \@minus -.2ex}%
                                     {1.5ex \@plus .2ex}%
                                     {\sf\normalsize\bfseries}}

\begin{document}

\title{Fortran Standard Activities}

\maketitle

\section{Organization of Standards Committees}

\subsection{ISO, SC22 and WG5}

The production and maintenance of International Standards in the field 
of Information Technology is carried out by the International Organization 
for Standardization (ISO) and the International Electrotechnical Commission 
(IEC) through their Joint Technical Committee 1 (JTC1). Most of the 
work of JTC1 is, however, delegated to one of 20 Sub-Committees (SCs) 
which, in turn, normally delegate the detailed work to permanent Working 
Groups (WGs).

SC22 is the JTC1 Sub-Committee responsible for \textit{programming languages, 
their environments and system software interfaces}, and carries out 
most of its work through its Working Groups. WG5 is the Working Group 
responsible for the development and maintenance of Fortran language 
standards.

WG5, like all other JTC1 Working Groups, consists of a Convenor and 
one or more Project Editors, all appointed by the parent Sub-Committee 
(SC22 in WG5's case), together with a number of individual members 
who are nominated by their National Member Bodies. It is not possible 
to apply directly to become an individual member of a Working Group 
without the support of an appropriate Member Body.

WG5 currently operates a two-tier development structure whereby the 
Working Group itself determines the broad technical content of the 
Standards and other formal documents that it is working on, but delegates 
much of the detailed work to \textit{development bodies} who are charged 
with assisting the Project Editor for a particular project to create 
the actual text for the document. For the Fortran 95, Fortran 2003, 
Fortran 2008 and Fortran 2018 Standards, the US Fortran Committee 
(INCITS/PL22.3) was given the role of Primary Development Body. All 
WG5 working documents are available on the WG5 Electronic Document 
Archive.

Their home page is:

\begin{verbatim}
https://wg5-fortran.org/
\end{verbatim}

\subsection{Upcoming Meetings} 

Links to agendas, local arrangements, etc. appear as the meeting dates 
approach.

\begin{itemize}

\item{The 2020, June 22-26, Minneapolis, Minnesota, USA, hosted by Bill Long, has been cancelled due to COVID-19. Details of the rescheduled meeting are given below.} 

\item{2020, October 12-16, Las Vegas, Nevada, USA.}

\item{2021, June 21-25, Manchester, UK. Host Nathan Sircombe}

\end{itemize}

Visit 

\begin{verbatim}
https://wg5-fortran.org/meetings.html
\end{verbatim}

for up to date information.

\subsection{Officers}

\begin{itemize}

\item{WG5 Convenor: Steve Lionel (US)}

\item{ISO/IEC 1539 Project Manager (formerly called Project Editor): Malcolm Cohen (UK)}

\item{Corrigenda Editor: David Muxworthy (UK)}

\end{itemize}


John Reid (UK) was WG5 Convenor from 1999 through 2017. From 1995 
until 1999, the Convenor of WG5 was Miles Ellis (UK). His predecessor 
was Jeanne Martin (USA), who was Convenor from 1982 until 1994. Prior 
to that Jeanne Adams (USA) was Convenor of WG5 and its predecessor, 
the Fortran Experts Group, from the creation of the latter in 1978 
until 1982.

The editor of the Fortran 2003 Standard (ISO/IEC 1539-1:2004(E)) was 
Richard Maine (USA).

The editor of the Fortran 95 Standard (ISO/IEC 1539-1:1997) was Richard 
Maine (USA).

The editor of the Fortran 90 Standard (IS 1539:1991) was Lloyd Campbell 
(USA) during most of its development and Mike Metcalf (CERN) during 
the final stages.


\subsection{Documents}

Documents are available at

\begin{verbatim}
https://wg5-fortran.org/documents.html
\end{verbatim}

This is an index of all WG5 documents, with links to those that are 
available in electronic form. They are in reverse order so that the 
most recent appear first. Electronic versions exists for most documents from 1995 to the current day. N003 is the earliest, and is dated 1978.

Where not otherwise indicated, papers were originated by the Convenor: 
up to N1074, Jeanne Martin; N1075 to N1374, Miles Ellis.

N001 to N099 were left free for pre-1985 documents and N100 to N199 
for 1985-1986 but they were never allocated in detail.

\subsection{Fortran 202x}

Fortran 202x is the working title of the revision of the Fortran standard 
after Fortran 2018. Previously it was referred to as Fortran 2020. 
Fortran 2018 is the current standard. 

This is the current development schedule (see N2135) for Fortran 202x. 
See the Glossary for definitions of terms.

\begin{itemize}

\item{Started planning further revision 2017-06}

\item{Choose issues that need attention 2018-06}

\item{Preliminary choice of technical content 2019-08}

\item{Final choice of technical content 2020-10}

\item{CD constructed 2021-06}

\item{CD ballot initiated 2021-07 }

\end{itemize}

\subsubsection{Selected Documents}

\begin{itemize}

\item{N2142 Fortran 2020 Feature Survey Results 2017-10 (Lionel)}

\end{itemize}


\subsection{PL22.3 - Programming Language Fortran}

PL22.3 - Programming Language Fortran, (formerly J3) is the US Fortran 
standards committee, a technical subcommittee of the International 
Committee for Information Technology Standards (INCITS) formerly known 
as the National Committee for Information Technology Standards (NCITS). 
The INCITS address is

\begin{verbatim}
http://www.incits.org/
\end{verbatim}

PL22 is the committee for Programming Languages. Their address is

\begin{verbatim}
http://www.incits.org/committees/pl22
\end{verbatim}

and PL22.3

\begin{verbatim}
http://www.incits.org/committees/pl22.3
\end{verbatim}

is the Fortran committee. Current participants, taken from the PL22.3 
site include

\begin{itemize}

\item{Bierman - Emeritus}

\item{Brainerd - Emeritus} 

\item{Clune - NASA}

\item{Hendrickson - Emeritus} 

\item{Hewlett Packard Enterprise - Bill Long}

\item{Hirchert - Emeritus} 

\item{IBM Corporation - Daniel Chen}

\item{Intel Corporation - Lorri Menard}

\item{Jet Propulsion Laboratory - Van Snyder}

\item{Klimowicz - NVIDIA}

\item{Lahey - Emeritus} 

\item{Maine - Emeritus} 

\item{Martin - Emeritus} 

\item{Meissner - Emeritus} 

\item{National Center for Atmospheric Research - Dan Nagle}

\item{North - Emeritus} 

\item{Oracle - Robert Corbett}

\item{Smith - Emeritus}

\item{The Numerical Algorithms Group Ltd - Malcolm Cohen}

\item{United States Dept of Energy}

\begin{itemize}

\item{Lawrence Berkeley National Lab}

\item{Los Alamos National Lab}

\item{Sandia National Lab}

\item{Oak Ridge National Lab}

\end{itemize}

\item{Wagener - Emeritus}

\end{itemize}

J3 developed the Fortran 66, Fortran 77, Fortran 90, Fortran 95, Fortran 
2003, Fortran 2008 and Fortran 2018 standards. Working closely with ISO/IEC/JTC1/SC22/WG5 
(see "WG5"), the international Fortran standards committee, J3 is 
the primary development body for Fortran. Fortran 2018 was published 
in 2018.

J3 is currently working under WG5 direction to produce a new revision 
of the Fortran standard, tentatively called Fortran 202x. This will 
be a minor revision of Fortran 2018. The work plan and schedule are 
available.

J3 meetings, documents, and membership are open to anyone worldwide. 
Meeting information and documents are available from this website; 
membership information may be obtained from any committee member. 
Any interested party can submit a paper for consideration at a J3 
meeting by following the J3 committee guidelines.

\subsubsection{PL22.3 Officers}

There are officially only 3 officers. 

\begin{itemize}

\item{Chair - Dan Nagle}

\item{Secretary - Lorri Menard}

\item{Treasurer - Jon Steidl}

\end{itemize}

The following list is taken from the J3 site.

\begin{itemize}

\item{Voting Principals}

\begin{itemize}

\item{Bryce Adelstein-Lelbach, Lawrence Berkeley National Laboratory}

\item{Daniel Chen, IBM Corp.}

\item{Thomas Clune, NASA GSFC}

\item{Robert Corbett, Robert Corbett (self)}

\item{Thomas Knox Kernelyze, LLC}

\item{Gary Klimowicz, Nvidia Corporation}

\item{Steve Lionel, Steve Lionel (Self) WG5 Convenor}

\item{William Long, HPE Inc.}

\item{Lorri Menard, Intel Corporation}

\item{Karla Morris, Sandia National Laboratories}

\item{Dan Nagle, Chair J3, National Center for Atmospheric Research}

\item{Craig E Rasmussen}

\item{Van Snyder, Jet Propulsion Laboratory}

\end{itemize}

\item{Voting alternatives}

\begin{itemize}

\item{Malcolm Cohen, Craig Rasmussen}

\item{Brian Friesen, Bryce Adelstein-Lebach}

\item{Andrew Gontarek, William Long}

\item{Henry Jin, Tom Clune}

\item{Mark LeAir, Gary Klimowicz}

\item{Kelvin Li, Daniel Chen}

\item{Raghu Maddhipatla, Lorri Menard}

\item{Divya Mangudi, Lorri Menard}

\item{Toon Moene, Karla Morris}

\item{John K. Reid, Dan Nagle}

\item{Damian Rouson, Karla Morris}

\item{Dr Anton Shterenlikht, William Long}

\item{Jon Steidel, Lorri Menard}

\item{Dr. John Wallin, Dan Nagle}

\item{Rafik Zurob, Daniel Chen}

\end{itemize}

\item{J3 INCITS Advisory members}

\begin{itemize}

\item{Lynn Barra}

\item{Deborah Spittle}

\end{itemize}

\item{J3 Liason}

\begin{itemize}

\item{Malcolm Cohen, WG5 Project Editor}

\item{David T. Muxworthy, WG5/BSI}

\item{Steve Lionel, WG5 Convenor}

\item{Minoru Tanaka, WG5/Japan}

\end{itemize}

\end{itemize}

\subsubsection{INCITS Secretariat}

Information Technology Industry Council, Secretariat, ITIC, Suite 
200, 1250 Eye Street NW, Washington DC 20005, Tel: (202) 737-8888

\subsubsection{PL22.3 Meetings}

PL22.3 currently plans three meetings per year, scheduled for the 
second week of February, June, and October. Meetings of J3 are open 
to the public, but facilities are limited. If you would like to attend 
a meeting, request further information from Dan Nagle.

\end{document}

